\documentclass[12pt,]{article}
\usepackage{lmodern}
\usepackage{amssymb,amsmath}
\usepackage{ifxetex,ifluatex}
\usepackage{fixltx2e} % provides \textsubscript
\ifnum 0\ifxetex 1\fi\ifluatex 1\fi=0 % if pdftex
  \usepackage[T1]{fontenc}
  \usepackage[utf8]{inputenc}
\else % if luatex or xelatex
  \ifxetex
    \usepackage{mathspec}
  \else
    \usepackage{fontspec}
  \fi
  \defaultfontfeatures{Ligatures=TeX,Scale=MatchLowercase}
\fi
% use upquote if available, for straight quotes in verbatim environments
\IfFileExists{upquote.sty}{\usepackage{upquote}}{}
% use microtype if available
\IfFileExists{microtype.sty}{%
\usepackage[]{microtype}
\UseMicrotypeSet[protrusion]{basicmath} % disable protrusion for tt fonts
}{}
\PassOptionsToPackage{hyphens}{url} % url is loaded by hyperref
\usepackage[unicode=true]{hyperref}
\hypersetup{
            pdfborder={0 0 0},
            breaklinks=true}
\urlstyle{same}  % don't use monospace font for urls
\usepackage[margin=3cm]{geometry}
\usepackage{longtable,booktabs}
% Fix footnotes in tables (requires footnote package)
\IfFileExists{footnote.sty}{\usepackage{footnote}\makesavenoteenv{long table}}{}
\IfFileExists{parskip.sty}{%
\usepackage{parskip}
}{% else
\setlength{\parindent}{0pt}
\setlength{\parskip}{6pt plus 2pt minus 1pt}
}
\setlength{\emergencystretch}{3em}  % prevent overfull lines
\providecommand{\tightlist}{%
  \setlength{\itemsep}{0pt}\setlength{\parskip}{0pt}}
\setcounter{secnumdepth}{0}
% Redefines (sub)paragraphs to behave more like sections
\ifx\paragraph\undefined\else
\let\oldparagraph\paragraph
\renewcommand{\paragraph}[1]{\oldparagraph{#1}\mbox{}}
\fi
\ifx\subparagraph\undefined\else
\let\oldsubparagraph\subparagraph
\renewcommand{\subparagraph}[1]{\oldsubparagraph{#1}\mbox{}}
\fi

% set default figure placement to htbp
\makeatletter
\def\fps@figure{htbp}
\makeatother


\date{}

\begin{document}

{
\setcounter{tocdepth}{3}
\tableofcontents
}
\section{System}\label{system}

\subsection{Dice}\label{dice}

The base dice used for the system is the \emph{d10}, or a 10-sided dice.
A percentile roll, the \emph{d100}, may be made by using 2d10, treating
the first roll as the value of the tenth digit, and the second roll as
the value of the ones digit. A 10 rolled for the tenth digit represents
a 0, unless both are rolled 10, in which case they collectively form a
100.

\subsection{Characteristic scores}\label{characteristic-scores}

\textbf{Characteristic scores} are key to a character's basic
construction. They are an abstraction of a character's basic abilities
and characteristics, of everything from their physical to mental
attributes, to weapon skills.

The following characteristics exist.

\subsubsection{Melee weapon strength}\label{melee-weapon-strength}

\emph{Abbreviated as MWS.}

\textbf{Melee weapon strength} represents a character's familiarity and
proficiency in utilising melee weapons.

All melee weapons that are not classified as \emph{Unarmed Melee} or
\emph{Brute Melee} utilise MWS rolls.

\subsubsection{Ranged weapon strength}\label{ranged-weapon-strength}

\emph{Abbreviated as RWS.}

\textbf{Ranged weapon strength} represents a character's familiarity and
proficiency in utilising ranged weapons.

All ranged weapons that are not classified as \emph{Unarmed Melee}
utilise RWS rolls.

\subsubsection{Strength}\label{strength}

\emph{Abbreviated as STR.}

\textbf{Strength} is a character's basic ability to achieve physical
feats. These are mostly those that involve feats of strength, such as
lifting, or carrying items, but also include acts such as climbing, and
jumping.

\emph{Unarmed Melee}, and \emph{Brute Melee} weapons also utilise STR
rolls.

\subsubsection{Agility}\label{agility}

\emph{Abbreviated as AGL.}

\textbf{Agility} represents a character's dexterity, physical
flexibility, and nimbleness. A character's initiative is also determined
by AGL.

\emph{Unarmed Ranged} weapons also utilise AGL rolls.

\subsubsection{Toughness}\label{toughness}

\emph{Abbreviated as TGH.}

\textbf{Toughness} represents a character's physical resilience and
natural defence capabilities. This covers a wide range of different
situations, from resisting status effects such as poison, to retaining
consciousness after sustaining damage, to surviving when potentially
lethal damage is taken.

\subsubsection{Mind}\label{mind}

\emph{Abbreviated as MND.}

\textbf{Mind} is a abstraction of mental processes and abilities. It
covers a character's intelligence, their ability to process and retain
information, and their mental capacity and willpower. MND is also used
when resisting mental status effects, and to resist damage to memory or
mental functions.

\subsubsection{Charisma}\label{charisma}

\emph{Abbreviated as CHA.}

\textbf{Charisma} is a character's ability to charm and coerce. This may
be derived from an innate ability to manipulate others, but may involve
other characteristics, including a character's physical attractiveness,
and a character's ability to speak publicly.

\subsubsection{Perception}\label{perception}

\emph{Abbreviated as PER.}

\textbf{Perception} is a character's ability to notice and perceive
their surroundings. This include complex uses of senses, the ability to
detect small details, and the ability to passively, and actively detect
other characters' actions.

\subsection{Checks}\label{checks}

\subsubsection{Basic \& characteristic
checks}\label{basic-characteristic-checks}

\textbf{Basic checks} are all made in a similar fashion. In order to
make a check, first determine a the check's difficulty. A basic,
challenging difficulty check is equal to the character's relevant
characteristic point. This is also known as a \textbf{characteristic
check}.

Characteristic checks are also made in other situations, such as when
resisting status effects, taking special actions in combat, and most
other circumstances when skills do not apply. Making these checks is the
same as making a basic check.

\emph{Check difficulties for basic and trained skill checks and their
corresponding modifiers. Please note modifiers in general are applied to
the difficulty of the check, not the roll.}

\begin{longtable}[]{@{}ll@{}}
\toprule
Difficulty & Modifier\tabularnewline
\midrule
\endhead
Trivial & +40\tabularnewline
Simple & +30\tabularnewline
Routine & +20\tabularnewline
Standard & +10\tabularnewline
Challenging & 0\tabularnewline
Hard & -10\tabularnewline
Very Hard & -20\tabularnewline
Painstaking & -30\tabularnewline
Impossible & -40\tabularnewline
\bottomrule
\end{longtable}

After a difficulty has been determined, roll a d100 and compare the roll
to the difficulty. If the roll is lower than the difficulty, the check
succeeds.

\subsubsection{Skill checks}\label{skill-checks}

\textbf{Skill checks} are made when a character utilises a known skill.
Making a skill check is the same as making a basic check, though the
player may choose which characteristic they wish to invoke. The results
and means through which an action is taken as a result may be affected
by the characteristic utilised.

\subsubsection{Untrained skill checks}\label{untrained-skill-checks}

\textbf{Untrained skill checks} are made when a character does not
possess a skill, but wishes to make a check untrained. In these
circumstances, the rules are the same for making a trained skill check,
but all modifies receive an additional -10.

\emph{Check difficulties for untrained skill checks.}

\begin{longtable}[]{@{}ll@{}}
\toprule
Difficulty & Modifier\tabularnewline
\midrule
\endhead
Trivial & +30\tabularnewline
Simple & +20\tabularnewline
Routine & +10\tabularnewline
Standard & 0\tabularnewline
Challenging & -10\tabularnewline
Hard & -20\tabularnewline
Very Hard & -30\tabularnewline
Painstaking & -40\tabularnewline
Impossible & -50\tabularnewline
\bottomrule
\end{longtable}

\subsubsection{Opposing checks}\label{opposing-checks}

\textbf{Opposing checks} are made when a character uses a skill or
characteristic against another character. In these situations, a basic
check is made for both characters with the applicable modifiers. The
character closest to success, or with the greatest success, succeeds the
check.

In an opposing check, it is impossible for both parties to fail or
succeed.

\subsection{Skills}\label{skills}

A list of skills in the basic system.

\begin{itemize}
\tightlist
\item
  Acrobatics
\item
  Animal Handling
\item
  Appraise
\item
  Arcana
\item
  Artificing
\item
  Athletics
\item
  Barter
\item
  Bluff
\item
  Bureaucracy
\item
  Business
\item
  Calligraphy
\item
  Charm
\item
  Climb
\item
  Contortion
\item
  Courting
\item
  Crafting
\item
  Culture
\item
  Deception
\item
  Deduce Motive
\item
  Detect Trap
\item
  Diplomacy
\item
  Disable Device
\item
  Disguise
\item
  Drawing
\item
  Ecology
\item
  Engineering
\item
  Escape
\item
  Etiquette
\item
  Forgery
\item
  Geography
\item
  History
\item
  Insight
\item
  Instruction
\item
  Intimidation
\item
  Investigation
\item
  Leadership
\item
  Linguistics
\item
  Lip Reading
\item
  Logic
\item
  Martial Arts
\item
  Medicine
\item
  Nature
\item
  Navigation
\item
  Painting
\item
  Penmanship
\item
  Performance
\item
  Persuasion
\item
  Philosophy
\item
  Politics
\item
  Psychology
\item
  Religion
\item
  Ride
\item
  Sail
\item
  Sleight of Hand
\item
  Smithing
\item
  Stealth
\item
  Strategy
\item
  Surgery
\item
  Survival
\item
  Swim
\item
  Tracking
\end{itemize}

\subsubsection{Skill progression}\label{skill-progression}

Skills have levels that go from 1-10. Each level above 1 provides an
additional +5 to checks against that skill. A basic training in a skill
provides a level 1 skill. In order to progress levels, upgrade points
(UP) may be spent.

1 UP should cost 1000 EXP.

\emph{Number of UP needed to progress from each level}

\begin{longtable}[]{@{}ll@{}}
\toprule
Level & Cost (UP)\tabularnewline
\midrule
\endhead
1 & 2\tabularnewline
2 & 4\tabularnewline
3 & 6\tabularnewline
4 & 12\tabularnewline
5 & 20\tabularnewline
6 & 32\tabularnewline
7 & 42\tabularnewline
8 & 84\tabularnewline
9 & 136\tabularnewline
\bottomrule
\end{longtable}

\section{Combat}\label{combat}

\subsection{Damage system}\label{damage-system}

The \textbf{Damage system} is designed to be challenging and deadly. It
utilises a damage model based on wounds, rather than hit points.

\subsubsection{Wounds}\label{wounds}

Each location tracks \textbf{wounds}. These represent physical damage
done to a location. Body locations include:

\begin{itemize}
\tightlist
\item
  Head
\item
  Right arm
\item
  Left arm
\item
  Torso
\item
  Right leg
\item
  Left leg
\end{itemize}

Each location may sustain three types of damage.

\begin{itemize}
\tightlist
\item
  \textbf{Flesh-wounds:} A hit that results in no immediate danger.
  However, little things build up, and one must be wary of these minor
  injuries. Four flesh-wounds form an injury.
\item
  \textbf{Injury:} A strike that is physically damaging. Each injury on
  a location grants a -5 modifier to utilising that location.
\item
  \textbf{Critical Injury:} An injury that disables the location
  entirely. If this is to the head or torso, it creates a mortal injury.

  \begin{itemize}
  \tightlist
  \item
    Upon receiving a critical injury, a character must make a TGH check.
    On failure, the location may be removed entirely. This does not
    apply to the head or torso.
  \end{itemize}
\end{itemize}

\emph{The number of injuries to each location needed to form a Critical
Injury on a normal human, or human-like creature.}

\begin{longtable}[]{@{}ll@{}}
\toprule
Region & Hits\tabularnewline
\midrule
\endhead
Arms & 10\tabularnewline
Legs & 10\tabularnewline
Head & 5\tabularnewline
Torso & 20\tabularnewline
\bottomrule
\end{longtable}

\emph{The dice rolls when hitting, and their corresponding regions.}

\begin{longtable}[]{@{}ll@{}}
\toprule
Region & Dice\tabularnewline
\midrule
\endhead
Head & 1-10\tabularnewline
Right arm & 11-20\tabularnewline
Left arm & 21-30\tabularnewline
Torso & 31-70\tabularnewline
Right leg & 71-85\tabularnewline
Left leg & 86-00\tabularnewline
\bottomrule
\end{longtable}

\subsubsection{Mortal Damage}\label{mortal-damage}

Once a character receives a \textbf{mortal injury}, they become unable
to function. If they are not stabilised within two turns, they must
start rolling toughness checks every turn. If at any point they fail
this check, they die.

\subsubsection{Weapons}\label{weapons}

Each \textbf{weapon} possesses a percentile chance of inflicting an
injury. This is represented by a weapon's DP, or Damage Percentile. If a
roll for damage beats (comes below) this value, an injury is inflicted.

If the roll for damage fails, a flesh-wound is inflicted instead.

If the roll surpasses four degrees of success, it automatically inflicts
a critical injury.

\subsubsection{Armour}\label{armour}

\textbf{Armour} is available for each location. A piece of armour will
possess an armour modifier. The armour modifier is added to the
difficulty when rolling to hit.

\subsection{Modifiers}\label{modifiers}

\subsubsection{Status Effects}\label{status-effects}

\textbf{Status effects} are additional effects applied onto characters.
These effects may be beneficial, or negative.

\begin{itemize}
\tightlist
\item
  Blindness
\item
  Charmed
\item
  Deafness
\item
  Fatigue
\item
  Fright
\item
  Incapacitation
\item
  Inspiration
\item
  Paralysis
\item
  Poison
\item
  Rush
\item
  Stun
\end{itemize}

\subsection{Combat Flow}\label{combat-flow}

\subsubsection{Initiating Combat}\label{initiating-combat}

When \textbf{initiating combat}, all character should roll initiative
equal to (x)d10, where x is equal to the tenth digit of the AGL score.

If the attack is unexpected by the defenders, each attacker receives (in
arbitrary order) one extra full turn before initiative is taken into
account.

\subsubsection{Turn actions}\label{turn-actions}

\textbf{Turns} are comprised of two stages, an action, and a move.
Unless otherwise specified, a non-movement action immediately ends the
turn.

\begin{itemize}
\tightlist
\item
  \textbf{Move:}

  \begin{itemize}
  \tightlist
  \item
    Move within the movement speed of the character.
  \end{itemize}
\item
  \textbf{Sprint:}

  \begin{itemize}
  \tightlist
  \item
    Move twice the movement speed of the character. This ends the turn.
  \end{itemize}
\item
  \textbf{Defend:}

  \begin{itemize}
  \tightlist
  \item
    Defend against the next attack. An attack the following turn has a
    -10 to hitting the defender.
  \end{itemize}
\item
  \textbf{Ranged / Melee Attack:}

  \begin{itemize}
  \tightlist
  \item
    Make the appropriate roll (MWS or RWS) to hit the enemy. The
    attacker may choose to make an aimed shot at a -20 to hit.
  \item
    Roll a d100 to determine the location hit. Refer to Locational
    Damage for chart.
  \item
    On success, make a roll against the weapon's DP for each of the
    weapon's speed. A weapon with 3 speed may roll damage 3 times.
  \item
    Repeat roll to hit and damage for each attack(s) per action.
  \end{itemize}
\item
  \textbf{Magic Attack:}

  \begin{itemize}
  \tightlist
  \item
    Make a magic roll based on the type of magic user.
  \item
    On success, apply spell effects to target. The target may make an
    agility roll to dodge (MOD -20) if their AGL score is above 20.
  \end{itemize}
\item
  \textbf{Other Actions:}

  \begin{itemize}
  \tightlist
  \item
    Use any other action that may be used outside of combat.
  \end{itemize}
\end{itemize}

\section{Classes}\label{classes}

\textbf{Classes} are key to a character's basic build. They may provide
skills, equipment, and additional features and abilities.

\subsection{Class Generalisation}\label{class-generalisation}

\subsubsection{Skills}\label{skills-1}

A class will typically provide training in \textbf{skills}.

\subsubsection{Equipment}\label{equipment}

A class will typically provide a set of \textbf{starting equipment}. The
item marked \emph{A knapsack with additional items and supplies.} should
be filled by the Game Master with items relevant to the setting or
adventure at hand.

\subsubsection{Class Feature}\label{class-feature}

A class will typically provide a \textbf{class feature}. A class feature
is an ability or bonus available to all users of that class from the
beginning.

\subsection{Channeller}\label{channeller}

\textbf{Skills:} Gain any five skills.

\textbf{Equipment:} Start with the following equipment:

\begin{enumerate}
\def\labelenumi{\arabic{enumi}.}
\tightlist
\item
  One piece of light armour \emph{or} A dagger.
\item
  A casting item.
\item
  A knapsack with additional items and supplies.
\end{enumerate}

\textbf{Class Feature:} You have with you the abilities of a lucky few.
The natural means to channel arcane energy through your body and form it
as you wish.

\subsection{Fighter}\label{fighter}

\textbf{Skills:} Gain four STR, TGH or AGL skills, one MDN or CHA skill,
and one PER skill.

\textbf{Equipment:} Start with the following equipment:

\begin{enumerate}
\def\labelenumi{\arabic{enumi}.}
\tightlist
\item
  One piece of medium armour \emph{or} One piece of light armour and any
  ranged weapon.
\item
  Any melee weapon and a shield \emph{or} Two melee weapons.
\item
  A knapsack with additional items and supplies.
\end{enumerate}

\textbf{Class Feature:} \emph{1)} Gain +10 in MWS, or \emph{2)} gain +10
RWS.

\subsection{Magician}\label{magician}

\textbf{Skills:} Gain any five skills.

\textbf{Equipment:} Start with the following equipment:

\begin{enumerate}
\def\labelenumi{\arabic{enumi}.}
\tightlist
\item
  One piece of light armour \emph{or} A dagger.
\item
  A casting item.
\item
  A knapsack with additional items and supplies.
\end{enumerate}

\textbf{Class Feature:} You have strived in your studies to unlock the
doors to the world of the arcane. Through your hard work, and sacrifice,
you have achieved your goals, and can cast the magic you have learned.

\subsection{None}\label{none}

\textbf{Skills:} Gain any six skills.

\textbf{Equipment:} Start with the following equipment:

\begin{enumerate}
\def\labelenumi{\arabic{enumi}.}
\tightlist
\item
  Any weapon.
\item
  Any armour.
\item
  A knapsack with additional items and supplies.
\end{enumerate}

\textbf{Class Feature:} Gain +8 in any characteristic.

\subsection{Rogue}\label{rogue}

\textbf{Skills:} Gain two AGL skills, two MDN or CHA skill, and two PER
skill.

\textbf{Equipment:} Start with the following equipment:

\begin{enumerate}
\def\labelenumi{\arabic{enumi}.}
\tightlist
\item
  One piece of light armour.
\item
  Any light melee weapon \emph{or} Any light ranged weapon.
\item
  A knapsack with additional items and supplies.
\end{enumerate}

\textbf{Class Feature:} Your past is well, a little dodgy to say the
least. You've almost certainly been involved in some sort of unsavoury
behaviour, but you've learned from these experiences as well. Gain
training in \emph{1)} Sleight of Hand, or \emph{2)} Stealth, or
\emph{13} +10 in AGL or PER.

\section{Equipment}\label{equipment-1}

\subsection{Weapon Classification}\label{weapon-classification}

\textbf{Weapons} are classified into a few broad categories. More
categories may be created depending on the needs of the world or story.

\subsubsection{Light Melee}\label{light-melee}

\textbf{Dagger}

\emph{Range:} 10m

\emph{DP:} 10

\emph{Speed:} 3

\subsubsection{Light Ranged}\label{light-ranged}

\subsubsection{Normal Melee}\label{normal-melee}

\textbf{Shortsword}

\emph{Range:} 0m

\emph{DP:} 40

\emph{Speed:} 2

\textbf{Longsword}

\emph{Range:} 0m

\emph{DP:} 60

\emph{Speed:} 1

\subsubsection{Normal Ranged}\label{normal-ranged}

\subsubsection{Unarmed Melee}\label{unarmed-melee}

\subsubsection{Unarmed Ranged}\label{unarmed-ranged}

\subsection{Armour Classification}\label{armour-classification}

Armour are classified into a few broad categories. More categories may
be created depending on the needs of the world or story.

\subsubsection{Light}\label{light}

\textbf{Leather}

\emph{Rating:} -15

\textbf{Gauntlet}

\emph{Rating:} -10

\subsubsection{Medium}\label{medium}

\subsubsection{Heavy}\label{heavy}

\subsubsection{Shields}\label{shields}

\section{Magic}\label{magic}

\subsection{Techniques}\label{techniques}

\textbf{Techniques} are verbs.

\textbf{Classical techniques} include the following:

\begin{itemize}
\tightlist
\item
  To command
\item
  To conjure
\item
  To infuse
\item
  To know
\item
  To mutate
\end{itemize}

\textbf{Free-form techniques} are any transitive action verbs.

\subsection{Aspects}\label{aspects}

\textbf{Aspects} are nouns.

\textbf{Classical elements} include the following:

\begin{itemize}
\tightlist
\item
  Aether
\item
  Air
\item
  Earth
\item
  Fire
\item
  Metal
\item
  Water
\item
  Wood
\end{itemize}

\textbf{Concrete aspects} are concrete nouns.

\textbf{Abstract aspects} are abstract nouns, in referring to ideas,
qualities, and conditions.

\textbf{Collective aspects} are collective nouns, in referring to a
collective of people or things.

\textbf{Proper aspects} are proper nouns.

\subsection{Forms}\label{forms}

Forms are always concrete nouns. They indicate the form in which the
aspect will be cast to.

\end{document}
