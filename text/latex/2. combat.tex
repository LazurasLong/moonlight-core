\documentclass[]{article}
\usepackage{lmodern}
\usepackage{amssymb,amsmath}
\usepackage{ifxetex,ifluatex}
\usepackage{fixltx2e} % provides \textsubscript
\ifnum 0\ifxetex 1\fi\ifluatex 1\fi=0 % if pdftex
  \usepackage[T1]{fontenc}
  \usepackage[utf8]{inputenc}
\else % if luatex or xelatex
  \ifxetex
    \usepackage{mathspec}
  \else
    \usepackage{fontspec}
  \fi
  \defaultfontfeatures{Ligatures=TeX,Scale=MatchLowercase}
\fi
% use upquote if available, for straight quotes in verbatim environments
\IfFileExists{upquote.sty}{\usepackage{upquote}}{}
% use microtype if available
\IfFileExists{microtype.sty}{%
\usepackage[]{microtype}
\UseMicrotypeSet[protrusion]{basicmath} % disable protrusion for tt fonts
}{}
\PassOptionsToPackage{hyphens}{url} % url is loaded by hyperref
\usepackage[unicode=true]{hyperref}
\hypersetup{
            pdfborder={0 0 0},
            breaklinks=true}
\urlstyle{same}  % don't use monospace font for urls
\usepackage[margin=1.2in]{geometry}
\usepackage{longtable,booktabs}
% Fix footnotes in tables (requires footnote package)
\IfFileExists{footnote.sty}{\usepackage{footnote}\makesavenoteenv{long table}}{}
\IfFileExists{parskip.sty}{%
\usepackage{parskip}
}{% else
\setlength{\parindent}{0pt}
\setlength{\parskip}{6pt plus 2pt minus 1pt}
}
\setlength{\emergencystretch}{3em}  % prevent overfull lines
\providecommand{\tightlist}{%
  \setlength{\itemsep}{0pt}\setlength{\parskip}{0pt}}
\setcounter{secnumdepth}{0}
% Redefines (sub)paragraphs to behave more like sections
\ifx\paragraph\undefined\else
\let\oldparagraph\paragraph
\renewcommand{\paragraph}[1]{\oldparagraph{#1}\mbox{}}
\fi
\ifx\subparagraph\undefined\else
\let\oldsubparagraph\subparagraph
\renewcommand{\subparagraph}[1]{\oldsubparagraph{#1}\mbox{}}
\fi

% set default figure placement to htbp
\makeatletter
\def\fps@figure{htbp}
\makeatother


\date{}

\begin{document}

\section{Combat}\label{combat}

Combat is designed to be challenging and deadly. It utilises a damage
model based on wounds, rather than hit points.

\subsection{Damage System}\label{damage-system}

\subsubsection{Wounds}\label{wounds}

Each location tracks wounds. Bodily locations include:

\begin{itemize}
\tightlist
\item
  Head
\item
  Right arm
\item
  Left arm
\item
  Torso
\item
  Right leg
\item
  Left leg
\end{itemize}

Each location may sustain three types of damage.

\begin{enumerate}
\def\labelenumi{\arabic{enumi}.}
\tightlist
\item
  \textbf{Flesh-wounds:} A hit that results in no immediate danger.
  However, they can build up. Three flesh-wounds create an injury.
\item
  \textbf{Injury:} A strike that is damaging. Each injury on a location
  grants a -5 modifier to utilising that location.
\item
  \textbf{Critical Injury:} An injury that disables the location
  entirely. If this is to the head or torso, it creates a mortal injury.

  \begin{itemize}
  \tightlist
  \item
    Upon receiving a critical injury, make a TGN check. On failure, the
    location is removed. This does not apply to the head or torso.
  \item
    The number of injuries to each location needed to form a Critical
    Injury on a normal human, or human-like creature:

    \begin{itemize}
    \tightlist
    \item
      Arms: 10
    \item
      Legs: 10
    \item
      Head: 5
    \item
      Torso: 20
    \end{itemize}
  \end{itemize}
\end{enumerate}

\subsubsection{Mortal Damage}\label{mortal-damage}

Once a character receives a mortal injury, they become unable to
function. If they are not stabilised within two turns, they must start
rolling toughness checks every turn. If at any point they fail this
check, they die.

\subsubsection{Weapons}\label{weapons}

Each weapon possesses a percentile chance of inflicting an injury. This
is represented by a weapon's DP, or Damage Percentile. If a roll for
damage beats (comes below) this value, an injury is inflicted.

If the roll for damage fails, a flesh-wound is inflicted instead.

If the roll surpasses four degrees of success, it automatically inflicts
a critical injury.

\subsubsection{Armour}\label{armour}

Armour is available for each location. A piece of armour will possess an
armour modifier. The armour modifier is added to the difficulty when
rolling to hit.

\subsubsection{Locational Damage}\label{locational-damage}

\begin{longtable}[]{@{}ll@{}}
\toprule
Region & Dice\tabularnewline
\midrule
\endhead
Head & 1-10\tabularnewline
Right arm & 11-20\tabularnewline
Left arm & 21-30\tabularnewline
Torso & 31-70\tabularnewline
Right leg & 71-85\tabularnewline
Left leg & 86-00\tabularnewline
\bottomrule
\end{longtable}

\subsection{Modifiers}\label{modifiers}

\subsubsection{Status Effects}\label{status-effects}

Status effects are additional effects applied onto characters. These
effects may be beneficial, or negative.

\begin{itemize}
\tightlist
\item
  Blindness
\item
  Charmed
\item
  Deafness
\item
  Fatigue
\item
  Fright
\item
  Incapacitation
\item
  Inspiration
\item
  Paralysis
\item
  Poison
\item
  Rush
\item
  Stun
\end{itemize}

\subsection{Combat Flow}\label{combat-flow}

\subsubsection{Initiating Combat}\label{initiating-combat}

When initiating combat, all character should roll initiative equal to
(x)d10, where x is equal to the tenth digit of the AGL score.

If the attack is unexpected by the defenders, each attacker receives (in
arbitrary order) one extra full turn before initiative is taken into
account.

\subsubsection{Turn actions}\label{turn-actions}

Turns are comprised of two actions. Unless otherwise specified, a
non-movement action immediately ends the turn.

\begin{itemize}
\tightlist
\item
  \textbf{Move:}

  \begin{itemize}
  \tightlist
  \item
    Move within the movement speed of the character.
  \end{itemize}
\item
  \textbf{Sprint:}

  \begin{itemize}
  \tightlist
  \item
    Move twice the movement speed of the character. This ends the turn.
  \end{itemize}
\item
  \textbf{Defend:}

  \begin{itemize}
  \tightlist
  \item
    Defend against the next attack. An attack the following turn has a
    -10 to hitting the defender.
  \end{itemize}
\item
  \textbf{Ranged / Melee Attack:}

  \begin{itemize}
  \tightlist
  \item
    Make the appropriate roll (MWS or RWS) to hit the enemy. The
    attacker may choose to make an aimed shot at a -20 to hit.
  \item
    Roll a d100 to determine the location hit. Refer to Locational
    Damage for chart.
  \item
    On success, make a roll against the weapon's DP.
  \item
    Repeat roll to hit and damage for each attack(s) per action.
  \end{itemize}
\item
  \textbf{Magic Attack:}

  \begin{itemize}
  \tightlist
  \item
    Make a magic roll based on the type of magic user.
  \item
    On success, apply spell effects to target. The target may make an
    agility roll to dodge (MOD -20) if their AGL score is above 30.
  \end{itemize}
\item
  \textbf{Other Actions:}

  \begin{itemize}
  \tightlist
  \item
    Use any other action that may be used outside of combat.
  \end{itemize}
\end{itemize}

\end{document}
