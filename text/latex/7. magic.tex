\documentclass[]{article}
\usepackage{lmodern}
\usepackage{amssymb,amsmath}
\usepackage{ifxetex,ifluatex}
\usepackage{fixltx2e} % provides \textsubscript
\ifnum 0\ifxetex 1\fi\ifluatex 1\fi=0 % if pdftex
  \usepackage[T1]{fontenc}
  \usepackage[utf8]{inputenc}
\else % if luatex or xelatex
  \ifxetex
    \usepackage{mathspec}
  \else
    \usepackage{fontspec}
  \fi
  \defaultfontfeatures{Ligatures=TeX,Scale=MatchLowercase}
\fi
% use upquote if available, for straight quotes in verbatim environments
\IfFileExists{upquote.sty}{\usepackage{upquote}}{}
% use microtype if available
\IfFileExists{microtype.sty}{%
\usepackage[]{microtype}
\UseMicrotypeSet[protrusion]{basicmath} % disable protrusion for tt fonts
}{}
\PassOptionsToPackage{hyphens}{url} % url is loaded by hyperref
\usepackage[unicode=true]{hyperref}
\hypersetup{
            pdfborder={0 0 0},
            breaklinks=true}
\urlstyle{same}  % don't use monospace font for urls
\usepackage[margin=1.2in]{geometry}
\usepackage{longtable,booktabs}
% Fix footnotes in tables (requires footnote package)
\IfFileExists{footnote.sty}{\usepackage{footnote}\makesavenoteenv{long table}}{}
\IfFileExists{parskip.sty}{%
\usepackage{parskip}
}{% else
\setlength{\parindent}{0pt}
\setlength{\parskip}{6pt plus 2pt minus 1pt}
}
\setlength{\emergencystretch}{3em}  % prevent overfull lines
\providecommand{\tightlist}{%
  \setlength{\itemsep}{0pt}\setlength{\parskip}{0pt}}
\setcounter{secnumdepth}{0}
% Redefines (sub)paragraphs to behave more like sections
\ifx\paragraph\undefined\else
\let\oldparagraph\paragraph
\renewcommand{\paragraph}[1]{\oldparagraph{#1}\mbox{}}
\fi
\ifx\subparagraph\undefined\else
\let\oldsubparagraph\subparagraph
\renewcommand{\subparagraph}[1]{\oldsubparagraph{#1}\mbox{}}
\fi

% set default figure placement to htbp
\makeatletter
\def\fps@figure{htbp}
\makeatother


\date{}

\begin{document}

\section{Magic}\label{magic}

The core ideas of magic operate on a single system. Magic spells are a
combination of a technique, an aspect, and a form.

\subsection{Techniques}\label{techniques}

\begin{itemize}
\tightlist
\item
  Commanding
\item
  Conjuring
\item
  Illusion*
\item
  Infusion
\item
  Invocation
\item
  Knowledge*
\item
  Mimic*
\item
  Mutation*
\item
  Protection
\end{itemize}

Techniques marked with an asterisk (*) do not need to invoke an aspect.

\subsection{Aspects}\label{aspects}

\begin{itemize}
\tightlist
\item
  Acid
\item
  Air
\item
  Arcane
\item
  Body
\item
  Celestial
\item
  Chaos
\item
  Dark
\item
  Death
\item
  Earth
\item
  Egg
\item
  Electricity
\item
  Fire
\item
  Force
\item
  Ghost
\item
  Glass
\item
  Gravity
\item
  Ice
\item
  Life
\item
  Light
\item
  Metal
\item
  Mind
\item
  Nature
\item
  Order
\item
  Poison
\item
  Sand
\item
  Sleep
\item
  Stone
\item
  Time
\item
  Vision
\item
  Water
\item
  Wood
\end{itemize}

\subsection{Forms}\label{forms}

\begin{itemize}
\tightlist
\item
  Arc
\item
  Aura
\item
  Beam
\item
  Being
\item
  Burst
\item
  Dispel
\item
  Entomb
\item
  Object
\item
  Projectile
\item
  Pure
\item
  Self
\end{itemize}

\subsection{Fluid Magic System}\label{fluid-magic-system}

The fluid magic system is dependent on a dynamic system of difficulty
and exhaustion.

\begin{longtable}[]{@{}ll@{}}
\toprule
Scale & Difficulty\tabularnewline
\midrule
\endhead
Inconsequential & 0\tabularnewline
Minor & 10\tabularnewline
Normal & 15\tabularnewline
Somewhat significant & 30\tabularnewline
Significant & 60\tabularnewline
Grand & 90\tabularnewline
Immense & 120\tabularnewline
Universal & 200\tabularnewline
\bottomrule
\end{longtable}

\begin{longtable}[]{@{}ll@{}}
\toprule
Technique & Difficulty\tabularnewline
\midrule
\endhead
Mutation & 10\tabularnewline
Invocation & 10\tabularnewline
Conjuring & 20\tabularnewline
Illusion & 20\tabularnewline
Mimic & 20\tabularnewline
Commanding & 30\tabularnewline
Protection & 30\tabularnewline
Infusion & 30\tabularnewline
Knowledge & 30\tabularnewline
\bottomrule
\end{longtable}

\begin{longtable}[]{@{}ll@{}}
\toprule
Level & Modifier\tabularnewline
\midrule
\endhead
Level 1 & +10\tabularnewline
Level 2 & +10\tabularnewline
Level 3 & +20\tabularnewline
Level 4 & 0\tabularnewline
Level 5 & 0\tabularnewline
Level 6 & 0\tabularnewline
Level 7 & 0\tabularnewline
Level 8 & 0\tabularnewline
Level 9 & 0\tabularnewline
Level 10 & 0\tabularnewline
Level 11 & 0\tabularnewline
Level 12 & -10\tabularnewline
Level 13 & -20\tabularnewline
Level 14 & -30\tabularnewline
Level 15 & -40\tabularnewline
Level 16 & -50\tabularnewline
Level 17 & -60\tabularnewline
Level 18 & -70\tabularnewline
Level 19 & -85\tabularnewline
Level 20 & -95\tabularnewline
\bottomrule
\end{longtable}

Level modifiers apply to both exhaustion and difficulty.

In order to calculate difficulty, add the base from the technique and
the scale, and any additional modifiers. A successful cast requires a
d100 roll that is lower than the total difficulty.

Speciality in a technique allows for a -20 to difficulty.

Exhaustion is equal to the difficulty squared divided by 70 (dif\^{}2 /
70) rounded to the nearest five.

Exhaustion decays at a rate of 10/hour.

\subsection{Rigid Magic System}\label{rigid-magic-system}

\subsection{Differences Between
Systems}\label{differences-between-systems}

There are two types of magic systems.

The core difference is that rigid spells allocate magical energy
beforehand, whereas fluid modify magical energy in real time. This is
particularly pertinent in situations when those who use the fluid system
(Channellers) use rigid (Magician) spells.

\end{document}
