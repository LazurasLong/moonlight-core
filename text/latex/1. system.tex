\documentclass[]{article}
\usepackage{lmodern}
\usepackage{amssymb,amsmath}
\usepackage{ifxetex,ifluatex}
\usepackage{fixltx2e} % provides \textsubscript
\ifnum 0\ifxetex 1\fi\ifluatex 1\fi=0 % if pdftex
  \usepackage[T1]{fontenc}
  \usepackage[utf8]{inputenc}
\else % if luatex or xelatex
  \ifxetex
    \usepackage{mathspec}
  \else
    \usepackage{fontspec}
  \fi
  \defaultfontfeatures{Ligatures=TeX,Scale=MatchLowercase}
\fi
% use upquote if available, for straight quotes in verbatim environments
\IfFileExists{upquote.sty}{\usepackage{upquote}}{}
% use microtype if available
\IfFileExists{microtype.sty}{%
\usepackage[]{microtype}
\UseMicrotypeSet[protrusion]{basicmath} % disable protrusion for tt fonts
}{}
\PassOptionsToPackage{hyphens}{url} % url is loaded by hyperref
\usepackage[unicode=true]{hyperref}
\hypersetup{
            pdfborder={0 0 0},
            breaklinks=true}
\urlstyle{same}  % don't use monospace font for urls
\usepackage[margin=1.2in]{geometry}
\usepackage{longtable,booktabs}
% Fix footnotes in tables (requires footnote package)
\IfFileExists{footnote.sty}{\usepackage{footnote}\makesavenoteenv{long table}}{}
\IfFileExists{parskip.sty}{%
\usepackage{parskip}
}{% else
\setlength{\parindent}{0pt}
\setlength{\parskip}{6pt plus 2pt minus 1pt}
}
\setlength{\emergencystretch}{3em}  % prevent overfull lines
\providecommand{\tightlist}{%
  \setlength{\itemsep}{0pt}\setlength{\parskip}{0pt}}
\setcounter{secnumdepth}{0}
% Redefines (sub)paragraphs to behave more like sections
\ifx\paragraph\undefined\else
\let\oldparagraph\paragraph
\renewcommand{\paragraph}[1]{\oldparagraph{#1}\mbox{}}
\fi
\ifx\subparagraph\undefined\else
\let\oldsubparagraph\subparagraph
\renewcommand{\subparagraph}[1]{\oldsubparagraph{#1}\mbox{}}
\fi

% set default figure placement to htbp
\makeatletter
\def\fps@figure{htbp}
\makeatother


\date{}

\begin{document}

\section{System}\label{system}

\subsection{Game Master}\label{game-master}

The person in charge of the game is known as the Game Master. This
system document assumes you are familiar to traditional tabletop
Role-playing games.

It must be stressed that the Game Master's word is final and absolute.
They are the person in charge of the game, and whatever they say counts
from a rules perspective. There will be parts of the game when a Game
Master's judgement is necessary.

\subsection{System guidelines and
functions}\label{system-guidelines-and-functions}

The system has a broad set of general rules-of-thumb to follow. All the
following rules should be taken as a guideline, and the Game Master's
word may override them.

\begin{enumerate}
\def\labelenumi{\arabic{enumi}.}
\item
  Specific rules take precedence over general rules. If there are
  contractions, a specific rule should be used over a general one.
\item
  When rounding is necessary, round downwards.
\end{enumerate}

\subsection{Dice}\label{dice}

The base dice used for the system is the \emph{d10}, or a 10-sided dice.
A percentile roll, the \emph{d100}, may be made by using 2d10, treating
the first roll as the value of the tenth digit, and the second roll as
the value of the ones digit.

\subsection{Characteristic scores}\label{characteristic-scores}

Characteristic scores are key to a character's basic construction. They
are an abstraction of a character's basic abilities and characteristics,
of everything from their physical to mental attributes, to weapon
skills.

The following characteristic scores exist.

\subsubsection{Strength}\label{strength}

\emph{Abbreviated as STR.}

\textbf{Strength} is a character's basic ability to achieve physical
feats. These are mostly those that involve feats of strength, such as
lifting, or carrying items, but also include acts such as climbing, and
jumping.

Unarmed melee, and brute melee weapons utilise STR rolls.

\subsubsection{Melee Weapon Strength}\label{melee-weapon-strength}

\emph{Abbreviated as MWS.}

\textbf{Melee weapon strength} represents a character's familiarity, and
proficiency in utilising complex melee weapons.

All non-unarmed, and brute melee weapons utilise MWS rolls.

\subsubsection{Agility}\label{agility}

\emph{Abbreviated as AGL.}

\textbf{Agility} represents a character's dexterity, physical
flexibility, and nimbleness. A character's initiative is also determined
by AGL.

Unarmed ranged weapons utilise AGL rolls.

\subsubsection{Ranged Weapon Strength}\label{ranged-weapon-strength}

\emph{Abbreviated as RWS.}

\textbf{Ranged weapon strength} represents a character's familiarity,
and proficiency in utilising complex ranged weapons.

All non-unarmed ranged weapons utilise RWS rolls.

\subsubsection{Toughness}\label{toughness}

\emph{Abbreviated as TGH.}

\textbf{Toughness} represents a character's physical resilience and
natural defence capabilities. This covers a wide range of different
situations, from resisting status effects such as poison, to retaining
consciousness after sustaining damage, to surviving when potentially
lethal damage is taken.

\subsubsection{Mind}\label{mind}

\emph{Abbreviated as MND.}

\textbf{Mind} is a abstraction of mental processes and abilities. It
covers a character's intelligence, their ability to process and retain
information, and their mental capacity and willpower. MND is also used
when resisting mental status effects, and to resist damage to memory or
mental functions.

\subsubsection{Charisma}\label{charisma}

\emph{Abbreviated as CHA.}

\textbf{Charisma} is a character's ability to charm and coerce. This may
be derived from an innate ability to manipulate, but may involve other
characteristics, including a character's physical attractiveness, and a
character's ability to speak.

\subsubsection{Perception}\label{perception}

\emph{Abbreviated as PER.}

\textbf{Perception} is a character's ability to notice and perceive
their surroundings. This include complex uses of senses, the ability to
detect small details, and the ability to passively, and actively detect
other characters' actions.

\subsection{Degrees of Success and
Failure}\label{degrees-of-success-and-failure}

A single degree of success is counted for every ten a d100 roll beats a
base difficulty. For example, if a check difficulty was 50, and the d100
roll was 30, we would say that roll had 2 degrees of success.

The degree of failure is similar, but counts every ten a d100 roll is
above a base difficulty. If a check difficulty was 50, and the d100 roll
was 72, we would say that roll had 2 degrees of failure.

\subsection{Checks}\label{checks}

Checks are split into two major types.

\subsubsection{Basic Checks}\label{basic-checks}

All basic checks are made in a similar fashion. In order to make a
check, first determine a the check's difficulty. A basic, challenging
difficulty check is equal to the character's relevant characteristic
point.

The following chart may be used to help determine the difficulty of
checks. Please note modifiers are applied to the difficulty of the
check, not the roll.

\begin{longtable}[]{@{}ll@{}}
\toprule
Difficulty & Modifier\tabularnewline
\midrule
\endhead
Trivial & +40\tabularnewline
Simple & +30\tabularnewline
Routine & +20\tabularnewline
Standard & +10\tabularnewline
Challenging & 0\tabularnewline
Hard & -10\tabularnewline
Very Hard & -20\tabularnewline
Painstaking & -30\tabularnewline
Impossible & -40\tabularnewline
\bottomrule
\end{longtable}

After a difficulty has been determined, roll a d100 and compare the roll
to the difficulty. If the roll is lower than the difficulty, the check
succeeds.

\subsubsection{Skill Checks}\label{skill-checks}

\textbf{Skill checks} are made when a character utilises a known skill.
Making a skill check is the same as making a basic check.

\subsubsection{Characteristic Checks}\label{characteristic-checks}

\textbf{Characteristic checks} are made using the base characteristic of
a character. The mechanics for characteristic checks vary depending on
the situation.

\paragraph{Untrained skill checks}\label{untrained-skill-checks}

Untrained skill checks are made when a character does not possess a
skill, but wishes to make a check untrained. In these circumstances, the
rules are the same for making a trained skill check, but all modifies
receive an additional -10.

\begin{longtable}[]{@{}ll@{}}
\toprule
Difficulty & Modifier\tabularnewline
\midrule
\endhead
Trivial & +30\tabularnewline
Simple & +20\tabularnewline
Routine & +10\tabularnewline
Standard & 0\tabularnewline
Challenging & -10\tabularnewline
Hard & -20\tabularnewline
Very Hard & -30\tabularnewline
Painstaking & -40\tabularnewline
Impossible & -50\tabularnewline
\bottomrule
\end{longtable}

\paragraph{Other Characteristic
Checks}\label{other-characteristic-checks}

Characteristic checks are also made in other situations, such as when
resisting status effects, rolling in combat, and most other
circumstances when skills do not apply. Making these checks is the same
as making a basic check.

\subsubsection{Opposing Checks}\label{opposing-checks}

An opposing check is made when a character uses a skill or
characteristic against another character. In these situations, a basic
check is made for both characters with the applicable modifiers. The
character with the least degrees of failure, or the most degrees of
success wins.

In an opposing check, it is impossible for both parties to fail, or
succeed.

\end{document}
